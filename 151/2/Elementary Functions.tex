% document configurations
\documentclass[letterpaper, 12pt]{article}
\usepackage[margin=1in]{geometry}

% document metadata
\newcommand{\headertitle}{Elementary Functions}
\newcommand{\weeknumber}{2}
\newcommand{\duedate}{2025/2/16}

% fancyhdr configurations
\usepackage{fancyhdr}
\pagestyle{fancy}
\setlength{\headheight}{15pt}
\fancyhead[L]{\bfseries\headertitle}
\fancyhead[R]{\bfseries Week \weeknumber: Due \duedate}

% amsthm configurations
\usepackage{amsthm}
\setcounter{section}{\weeknumber}
\theoremstyle{plain}
\newtheorem{theorem}{Theorem}[section]
\theoremstyle{definition}
\newtheorem{definition}[theorem]{Definition}
\newtheorem{problem}[theorem]{Problem}
\newtheorem*{reading}{Reading}
\newtheorem*{recall}{Recall}


\begin{document}

\begin{reading}
    Section 1.5-1.7 from \textit{Calculus: One and Several Variables}.
\end{reading}

Recall from week 1 where we introduced \textit{if-then} statements and \textit{iff} statements. Here we will give a more rigorous definition of \textit{if-then} statements, and use the definition to discuss and edge case of the \textit{if-then} statement where the conditional is always false.

\begin{definition}
    Statement \textit{If A then B} is equivalent to \textit{Not(A and Not(B))}.
\end{definition}

\begin{theorem}\label{Dummy}
    If it is raining, then the ground is wet.
\end{theorem}

\begin{problem}
    Write \ref{Dummy} in its equivalent form, using everyday language.\\
    \textit{Hint: Not() can be written as 'It can't be the case that\dots'}
\end{problem}

\begin{theorem}\label{True}
    If 1+1=3, then triangles have 4 sides.
\end{theorem}

\begin{problem}
    Denote $A$ as \textit{1+1=3}, and $B$ as \textit{triangles have 4 sides}.
    \begin{enumerate}
        \item Write \textit{Not(B)}. Is it true?
        \item Write \textit{A and Not(B)}. Is it true?
        \item Write \ref{True} using without using the \textit{if-then} format. Is it true?
    \end{enumerate}
\end{problem}

\end{document}